\documentclass{beamer}
\usetheme{Madrid}
%\usecolortheme{beaver}
\usepackage{listings}
\lstset{language=C++,
	backgroundcolor=\color{blue!50!gray!20!white},
	basicstyle=\ttfamily\tiny,
	keywordstyle=\color{black!40!blue}\bfseries,
	commentstyle=\color{black!40!green},
	tabsize=4}
  
\title{Concepts}
\author{Dimitris Bogiokas}
\date{Binary Spaces}

\begin{document}
	\maketitle

\begin{frame}[fragile]{}
\lstset{basicstyle=\small, backgroundcolor=\color{white}}
\begin{lstlisting}
template<class Person, typename = enable_if_t<
	can_parse_this_line_of_code<Person>::value>>
\end{lstlisting}
\vspace{-3.3em}
\maketitle
\end{frame}

\begin{frame}{Outline}
\tableofcontents
\end{frame}

\section{Template Metaprogramming}
\subsection{It's just different syntax}
\begin{frame}{Template Metaprogramming}
\begin{quote}
A metaprogram is a program that generates or manipulates program code. Ever since generic programming was introduced to C++, programmers have discovered myriad "template tricks" for manipulating programs as they are compiled, effectively eliminating the barrier between program and metaprogram.
\end{quote}
\begin{flushright}
C++ Template Metaprogramming:\\Concepts, Tools, and Techniques\\from Boost and Beyond
\end{flushright}
\end{frame}

\begin{frame}[fragile]{Compile time ``functions'', having weird syntax.}
	We think of the struct templates as compile time functions. Their ``input'' is the template parameters pack and their ``output'' is usually one of the following:
	\begin{itemize}
		\item Some \texttt{public static const} member of this struct.
\begin{lstlisting}
template<size_t M, size_t N>
struct min
{
	static constexpr size_t value = M <= N ? M : N;
};
\end{lstlisting}
		\item Some \texttt{public typedef} in this struct.
\begin{lstlisting}
template<typename T>
struct make_const
{
	using type = const T;
};
\end{lstlisting}
	\end{itemize}
We will use these conventions through the talk. Notice that the compiler cannot automatically deduce that \texttt{struct\_name<T>::type} is a typename, so we have to preface it with the ``\texttt{typename}'' keyword when using it.
\end{frame}

\begin{frame}[fragile]{Template specialization}
\begin{lstlisting}
template<size_t M, size_t N>
struct gcd
{
	static constexpr size_t value = gcd<N, M%N>::value;
};

template<size_t M>
struct gcd<M, 0u>
{
	static_assert(M != 0u);
	static constexpr size_t value = M;
};

int main()
{
	return gcd<120u, 420u>::value; // = 60u
}
\end{lstlisting}
\end{frame}

\begin{frame}[fragile]{What about \texttt{constexpr} functions?}
\begin{lstlisting}
constexpr size_t gcd(const size_t m, const size_t n)
{
	if(n == 0u)
		return m;
	else
		return gcd(n, m%n);
}

int main()
{
	return gcd(120u, 420u); // = 60u
}
\end{lstlisting}
\pause
\begin{lstlisting}
consteval size_t gcd(const size_t m, const size_t n) // c++20
{
	if(n == 0u)
	{
		if(m == 0u)
			throw char{};
		return m;
	}
	else
	{
		return gcd(n, m%n);
	}
}
\end{lstlisting}

\end{frame}

\begin{frame}[fragile]{Input can also be a type}
\begin{lstlisting}
template<class T>
struct rank
{
	static constexpr size_t value = 0u;
};

template<class U, size_t N>
struct rank<U[N]>
{
	static constexpr size_t value = 1u + rank<U>::value;
};


int main()
{
	return rank<int[2][3][1][5][3]>::value; // = 5u
}
\end{lstlisting}
\end{frame}

\begin{frame}[fragile]{Input and Output both can be types}
\begin{lstlisting}
template<class T>
struct remove_const
{
	using type = T;
};

template<class T>
struct remove_const<T const>
{
	using type = T;
};


int main()
{
	remove_const<const size_t>::type bar = 6u;
	++bar;
	return bar; // = 7u
}
\end{lstlisting}
\end{frame}

\begin{frame}[fragile]{Using inheritance as function branches}
\begin{lstlisting}
template<class T>
struct type_is { using type = T; };


template<class T>
struct remove_const : type_is<T> {};

template<class T>
struct remove_const<T const>
{
	using type = T;
};


template<class T>
struct remove_volatile : type_is<T> {};

template<class T>
struct remove_volatile<T volatile>
{
	using type = T;
};
\end{lstlisting}
\end{frame}

\begin{frame}[fragile]{\texttt{std::conditional}}
\begin{lstlisting}
template<class T> struct type_is { using type = T; };


template<bool, class T, class>
struct conditional : type_is<T> {};

template<class T, class F>
struct conditional<false, T, F> : type_is<F> {};


int main()
{
	conditional<true, int, size_t>::type foo = 0;
	std::cout<<--foo<<std::endl; // -1 or MAX, depending on the bool
	return foo;
}
\end{lstlisting}
\end{frame}

\begin{frame}[fragile]{A glimpse inside \texttt{type\_traits} library}
\begin{lstlisting}
#include<type_traits>
template<bool b> using bool_constant = std::integral_constant<bool, b>;
using true_type = bool_constant<true>;
using false_type = bool_constant<false>;

template<class T> struct is_void : false_type {};
template<> struct is_void<void> : true_type {};
template<> struct is_void<void const> : true_type {};
template<> struct is_void<void volatile> : true_type {};
template<> struct is_void<void const volatile> : true_type {};

template<class T, class U> struct is_same : false_type {};
template<class T> struct is_same<T, T> : true_type {};
//template<class T> using is_void = is_same<typename remove_cv<T>::type, void>;

int main()
{
	return is_void<int>::value; // = 0
}
\end{lstlisting}
The following aliases are used throughout the standard library for convenience:
\begin{lstlisting}
template<class...Pack>
struct struct_name;

template<class...Pack>
using struct_name_t = typename struct_name<Pack...>::type;

template<class...Pack>
using struct_name_v = struct_name<Pack...>::value;
\end{lstlisting}
\end{frame}

\begin{frame}[fragile]{Parameter pack example}
\begin{lstlisting}
#include<type_traits>
template<bool b> using bool_constant = std::integral_constant<bool, b>;
using true_type = bool_constant<true>;
using false_type = bool_constant<false>;

template<class T, class... P0toN>
struct is_one_of;

template<class T>
struct is_one_of<T> : false_type {};

template<class T, class... P1toN>
struct is_one_of<T, T, P1toN...> : true_type {};

template<class T, class P0, class... P1toN>
struct is_one_of<T, P0, P1toN...> : is_one_of<T, P1toN...> {};

int main()
{
	return is_one_of<float, char, int, long, double>::value; // = 0
}
\end{lstlisting}
\end{frame}

\subsection{SFINAE}
\begin{frame}[fragile]{Substitution Failure Is Not An Error}{or ``explicit overload set management''}
\begin{lstlisting}
template<class T> struct type_is { using type = T; };


template<bool, class T>
struct enable_if : type_is<T> {};

template<class T>
struct enable_if<false, T> {};


int main()
{
	enable_if<false, size_t>::type foo = 5; // doesn't compile
	return foo;
}
\end{lstlisting}
\end{frame}

\begin{frame}[fragile]{But why use SFINAE?}
\begin{lstlisting}
#include<type_traits>
template<class T> struct type_is { using type = T; };


template<bool, class T>
struct enable_if : type_is<T> {};

template<class T>
struct enable_if<false, T> {};

// size_t f(T x)
template<class T>
typename enable_if<std::is_integral<T>::value, size_t>::type f(T x)
{
	return x + 1;
}

// long double f(T x)
template<class T>
typename enable_if<std::is_floating_point<T>::value, long double>::type f(T x)
{
	return x - 1.0;
}

int main()
{
	return f(6); // = 7, whereas f(6.0) = 5.0
}
\end{lstlisting}
\end{frame}

\begin{frame}[fragile]{Not convinced yet?}
\begin{lstlisting}
#include<type_traits>

template<class T>
struct has_element_named_type
{
private:
	template<class U, class = U::type>
	static std::true_type test_f(U&&);
	
	static std::false_type test_f(...);
	
public:
	static constexpr bool value = decltype(test_f(std::declval<T>()))::value;
};

struct S
{
	using type = void;
};

int main()
{
	return has_element_named_type<S>::value; // = 1
}
\end{lstlisting}
\end{frame}

\begin{frame}[fragile]{\texttt{declval()} and other black magic functions}
These 4 functions can be called in an unevaluated context. I.e. they can help deduce information about some function's type, without ever calling the function:\\
\texttt{typeid, sizeof, noexcept, decltype}
\begin{lstlisting}
#include<type_traits>

double f(int x); // never defined anywhere

int main()
{
	decltype(f(std::declval<int>())) y; //compiles
	return 0;
}
\end{lstlisting}
\end{frame}

\begin{frame}[fragile]{Asking more difficult questions}
\begin{lstlisting}
#include<type_traits>

template<class T>
struct is_copy_assignable
{
private:
	template<class U,
		class = decltype(std::declval<U>().operator=(std::declval<U const&>()))>
	static std::true_type test_f(U&&);
	
	static std::false_type test_f(...);

public:
	static constexpr bool value = decltype(test_f(std::declval<T>()))::value;
};

struct S
{
	S& operator=(const S&) = delete;
};

int main()
{
	return is_copy_assignable<S>::value; // = 0
}
\end{lstlisting}
\end{frame}

\subsection{\texttt{void\_t}}
\begin{frame}[fragile]{The most trivial definition...}
\begin{lstlisting}
template<class...>
	using void_t = void;
\end{lstlisting}
\pause
Key feature: It vanishes if any of the template parameters is ill-formed, otherwise it always returns \texttt{void}.
\begin{lstlisting}
#include<type_traits>
template<class...> using void_t = void;

template<class, class = void> // it has to match with the definition of void_t
struct has_type_member : std::false_type {};

template<class T>
struct has_type_member<T, void_t<typename T::type>> : std::true_type {};

struct S
{
	using type = int;
};

int main()
{
	return has_type_member<S>::value;
}
\end{lstlisting}
\end{frame}

\begin{frame}[fragile]{Revisiting \texttt{is\_copy\_assignable}}
\begin{lstlisting}
#include<type_traits>
template<class T>
using copy_test_type = decltype(std::declval<T&>().operator=(std::declval<T const&>()));

template<class, class=void>
struct is_copy_assignable : std::false_type {};

template<class T>
struct is_copy_assignable<T, std::void_t<copy_test_type<T>>>
						  : std::is_same<copy_test_type<T>, T&> {}; //std::true_type

struct S
{
	S& operator=(const S&) = delete;
};

int main()
{
	return is_copy_assignable<S>::value;
}
\end{lstlisting}
\pause
This tool can now easily change to answer all kinds of questions.
\begin{lstlisting}
template<class T>
using move_test_type = decltype(std::declval<T&>().operator=(std::declval<T&&>()));
\end{lstlisting}
\pause
\begin{lstlisting}
template<class T>
using HelloWorld_test_type = decltype(std::declval<T&>().HelloWorld(std::declval<int>()));
\end{lstlisting}
\end{frame}

\section{Concepts}
\begin{frame}[fragile]{First concept}
\begin{lstlisting}
#include<type_traits>

template<typename T>
concept Negatable = std::is_signed_v<T>;

template<typename T> requires Negatable<T>
T negate(T input)
{
	return -input;
}

int main() { return negate(2u); } // doesn't compile
\end{lstlisting}
\pause
\begin{lstlisting}
template<typename T>
T negate(T input) requires Negatable<T>
{
	return -input;
}
\end{lstlisting}
\pause
\begin{lstlisting}
template<Negatable T>
T negate(T input)
{
	return -input;
}
\end{lstlisting}
\pause
\begin{lstlisting}
auto negate(Negatable auto input)
{
	return -input;
}
\end{lstlisting}
\end{frame}

\begin{frame}[fragile]{Solving the very first problem}
\begin{lstlisting}
#include<type_traits>

template<class T>
concept floating_point = std::is_floating_point_v<T>;

template<class T>
concept integral = std::is_integral_v<T>;

auto handleNumber(floating_point auto input)
{
	return input - 1.0;
}

auto handleNumber(integral auto input)
{
	return input + 1;
}

int main()
{
	return handleNumber(5u);
}
\end{lstlisting}
\end{frame}

\begin{frame}[fragile]{useful concept syntax}
\begin{lstlisting}
#include<type_traits>

template<class T>
concept floating_point = std::is_floating_point_v<T>;
template<class T>
concept integral = std::is_integral_v<T>;

template<typename T> requires floating_point<T> || integral<T>
auto handleNumber(T input)
{
	return input + 1;
}

int main()
{
	return handleNumber(5u);
}
\end{lstlisting}
\pause
\begin{lstlisting}
template<class T>
concept floating_point = std::is_floating_point_v<T>;
template<class T>
concept integral = std::is_integral_v<T>;

template<class T>
concept interesting_number = floating_point<T> || integral<T>;

auto handleNumber(interesting_number auto input)
{
	return input + 1.0;
}
\end{lstlisting}
\end{frame}

\begin{frame}[fragile]{Other concepts}
\begin{lstlisting}
#include<type_traits>
	
template<class T>
concept addable = requires(T x, T y) {
	x + y;
};

template<class T>
concept type_having = requires {
	typename T::type;
};

template<class T>
concept boolean = std::is_convertible_v<T, bool>;

template<class T>
concept equality_comparable = requires(const T& x, const T& y) {
	{ x == y } -> boolean;
	{ x != y } -> boolean;
};

template<class T>
concept partially_orderable = requires(const T& x, const T& y) {
	{ x < y } -> boolean;
	{ x > y } -> boolean;
	{ x <= y } -> boolean;
	{ x >= y } -> boolean;
};

template<class T>
concept totally_orderable = partially_orderable<T> && equality_comparable<T>;
\end{lstlisting}
\end{frame}


\begin{frame}[fragile]{Sources}
\href{https://www.youtube.com/watch?v=Am2is2QCvxY&ab_channel=CppCon}{CppCon 2014: Walter E. Brown\\``Modern Template Metaprogramming: A Compendium, Part I''}\\
\vspace{1em}
\href{https://www.youtube.com/watch?v=a0FliKwcwXE&ab_channel=CppCon}{CppCon 2014: Walter E. Brown\\``Modern Template Metaprogramming: A Compendium, Part II''}\\
\vspace{1em}
\href{https://www.youtube.com/watch?v=tiAVWcjIF6o&ab_channel=CppCon}{CppCon 2020: Jody Hagins\\``Template Metaprogramming: Type Traits (part 1 of 2)''}\\
\vspace{1em}
\href{https://www.youtube.com/watch?v=dLZcocFOb5Q&ab_channel=CppCon}{CppCon 2020: Jody Hagins\\``Template Metaprogramming: Type Traits (part 2 of 2)''}\\
\vspace{1em}
\href{https://www.youtube.com/watch?v=qawSiMIXtE4&ab_channel=CppCon}{CppCon 2019: Saar Raz\\``C++20 Concepts: A Day in the Life''}
\vspace{2em}
\pause
\lstset{basicstyle=\large, backgroundcolor=\color{white}}
\begin{lstlisting}
		Thank you!
\end{lstlisting}
\end{frame}
\end{document}

